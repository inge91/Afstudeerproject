
\documentclass{article}
\usepackage[utf8]{inputenc}
\usepackage[ngerman]{babel}

\title{Tender for Recognition of Touch Emotions}

\author{Inge Becht \\6093906}
\date{\today}

\begin{document}

\maketitle
Project: Recognition of Touch Emotions\\
Supervisors: Ben Kr"ose, Gwenn Englebienne\\

I have read your paper about the concept of TaSST and am interested in joining the research for this project. It would seem like a great opportunity to spend time on further development of a system that will be used outside of the thesis, and could bring some interesting information to light on how humans interact on a physical level.

The way I would possibly go about finding prototypes for certain emotions is by using unsupervised machine learning techniques. In case a large amount of data is available, one could look at which features stand out the most for a particular emotion, and which are the most reliable for a given group. Features could be elements like pressure, the surface area of the touch, the duration of the touch and such (as all these things are probably easily measurable using the TaSST). The importance and reliability of these features depends on how distinct they are from other emotions and how significant they are for an emotion itself. The features used could be made into a classifier, which could be tested on new data to see how well the prototypes were made and distinguishable from one another. If not, possibly more features can be considered, or a refinement of the values a certain feature might have.
As long as there is enough data and the TaSST is able to distinguish between enough of these features it should be possible to create a prototype for specific emotional actions. 

I consider myself a good candidate for this thesis due to my interest in the subject. I also  have taken different courses regarding machine learning which I think have taught me valuable tools for this particular project and which I completed with good grades. One of these courses was an open ended project in which large amount of data needed to be classified by determining the importance of all the available features in a matter similar to this project.
\end{document}
