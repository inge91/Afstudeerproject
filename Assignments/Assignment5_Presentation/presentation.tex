\documentclass{beamer}
\usetheme{Madrid} % My favorite!
\usepackage{color}
%\usetheme{Boadilla} % Pretty neat, soft color.
%\usetheme{default}
%\usetheme{Warsaw}
%\usetheme{Bergen} % This template has nagivation on the left
%\usetheme{Frankfurt} % Similar to the default 
%with an extra region at the top.
\usecolortheme{orchid} % Simple and clean template
%\usetheme{Darmstadt} % not so good
% Uncomment the following line if you want %
% page numbers and using Warsaw theme%
% \setbeamertemplate{footline}[page number]
%\setbeamercovered{transparent}
\setbeamercovered{invisible}
% To remove the navigation symbols from 
% the bottom of slides%
\setbeamertemplate{navigation symbols}{} 
%
\usepackage{graphicx}
%\usepackage{bm}         % For typesetting bold math (not \mathbold)
%\logo{\includegraphics[height=0.6cm]{yourlogo.eps}}
%
\title[Action Model Reasoner]{Making an Action Model Reasoner
    For a Game of Capture
the Flag}
\author{Inge Becht}
\institute[University of Amsterdam]
{
University of Amsterdam \\
\medskip
{}
}
\date{\today}
% \today will show current date. 
% Alternatively, you can specify a date.
%
\begin{document}
%
\begin{frame}
    \titlepage
\end{frame}

%FIXME Too many slides for a 8 minute presentation?
\begin{frame}
    \frametitle{Introduction}
    \begin{itemize}
        \item{Game AI research:}
            \begin{itemize}
                \item human-like
                \item {\color{red}fairness}
                    \begin{itemize}
                        \item incomplete knowledge
                    \end{itemize}
                \item adaptive
            \end{itemize}
    \end{itemize}
    \begin{block}{Problem statement}
        %FIXME is this too global? Should I add ME IRL to the statement
        Is it possible to \emph{successfully} build a motion model for a game of Capture the Flag that predicts enemy
        position and can this \emph{improve} the performance of an AI?
    \end{block}
\begin{itemize}
    \item Two aspects to this research
\end{itemize}
\end{frame}

\begin{frame}
    \frametitle{Tools}
    \begin{itemize}
        \item{Capture the Flag (CTF)}
            \begin{itemize}
                \item Team multi-agent environment
                \item Single target game
            \end{itemize}
        \item{AISandbox}
            \begin{itemize}
                \item CTF AI development environment
                \item Used for competitions
            \end{itemize}
        \item{Use pre-existing AI}
            \begin{itemize}
                \item Terminator: 3$^{rd}$ place in competition
                \item naive opponent response reasoner
                \item add own motion model maker
            \end{itemize}
    \end{itemize}
\end{frame}

\begin{frame}
\frametitle{Approach Motion Modelling}
\begin{itemize}
    \item Maximum Entropy Inverse Reinforcement Learning
    \begin{itemize}
        \item Feature based Reasoning
        \item Learns offline given known trajectories
        \item Distribution of occupation likelihood
    \end{itemize}
    %FIXME: Should i use references?
    \item Has been done, but:
        \begin{itemize}
            \item Different gaming domain (Non-team Death match)
            \item Uses one model per human player
            \item Only tested with interception reasoner
            \item Doesn't check map invariance
        \end {itemize}
\end{itemize}
\end{frame}

% FIXME Scientific enough?
\begin{frame}
\frametitle{Approach Reasoner}
\begin{itemize}
    \item What to do with new information?
    \item Different aspects:
    \begin{itemize}
        \item Ambushing enemy
        \item Avoiding enemy
        \item Anticipating actions
    \end{itemize}
    \item broadness of implementation depends on time left
    \item Finite state automaton
\end{itemize}
\end{frame}


\begin{frame}
\frametitle{Evaluation}
\begin{block}{Problem statement (again)}
Is it possible to \emph{successfully} build a motion model for a game of Capture the Flag that predicts enemy
position and can this \emph{improve} the performance of an AI?
\end{block}
\begin{itemize}
    \item Evaluation contains two aspects
    \item \emph{Successfully} build a motion model:
        \begin{itemize}
            \item absolute error
            \item possibly compare with other models
        \end{itemize}
    % Is this an honest measure of improvement?
    \item \emph{Improving} performance of an AI:
        \begin{itemize}
            \item Create Competition
            \item Depends on success of motion model.
        \end{itemize}
\end{itemize}
\end{frame}

\begin{frame}
\frametitle{Plan}
\begin{table}
\centering
    \begin{tabular}{| l | l |}
      \hline                        
      Week No. & Planning \\
      \hline
      \hline
      18 &  Implement IRL motion model \\
      \hline
      19 &  Implement IRL motion model \\
      \hline
      20 &  Implement model in AI code and work on reasoning\\
      \hline
      21 &  Work on reasoning aspect \\
      \hline
      22 &  Preparation midpresentation and assignment 8\\
      \hline
      23 &  Starting with evaluation\\
      \hline
      24 &  Assignment 9 and evaluation completion \\
      \hline
      25 &  Finishing paper \\
      \hline
      26 &  Preparing final presentation, finishing logbook and paper \\
      \hline
    \end{tabular}
\end{table}
\end{frame}



% End of slides
\end{document} 
